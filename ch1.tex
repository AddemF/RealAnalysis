\documentclass{article}

\usepackage{amsmath, amssymb, amsfonts}

\begin{document}
	\begin{center}
		\LARGE BABY RUDIN CHAPTER 1
	\end{center}

	{\Large Problem 1

	 If $r\in\mathbb{I}=\mathbb{R}\setminus \mathbb{Q}$ and $x\in\mathbb{Q}$ prove that $r+x, rx \in\mathbb{I}$.}

	By hypothesis there are no integers $a,b$ such that $r=a/b$, but there are integers such that $x=a/b$.  In particular let's say $x=a/b$ where $b\ne 0$ and $gcd(a,b)=1$.  For contradiction suppose $r+x = p/q$ for integers $p,q$.  This would imply 
	
	\begin{align*}
		r = p/q - x = \frac{pb-aq}{bq} \in \mathbb{Q}
	\end{align*}
	
	Also for contradiction assume $rx\in \mathbb{Q}$ and $rx = p/q$ to lazily re-use our old symbols.  Then so long as $a\ne 0$ we have
	
	\begin{align*}
		r = \frac{p}{q}\frac{1}{x} = \frac{pb}{qa} \in \mathbb{Q}
	\end{align*}
	
	and if $a=0$ then of course $rx = ra/b = 0\in\mathbb{Q}$.
	
	\pagebreak

	{\Large Problem 2}
	
	
\end{document}
